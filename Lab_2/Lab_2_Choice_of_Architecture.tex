\documentclass{article}
\usepackage[utf8]{inputenc}

\title{Lab_2_Choice_of_Architecture}
\author{wadedownton}
\date{August 2017}

\begin{document}
	
	\title{COMS3002 - Software Engineering}
	\author{Wade Downton (1171712)}
	\maketitle
	
	\section{ii) Choice of Architecture}
	\begin{enumerate}
		\item \textbf{Types of architectures and their pros and cons:}
		\newline
		
		
		
		\textbf{- Blackboard}
		\newline\newline
		\textbf{Description:}
		
		\begin{flushleft}
			A blackboard system is an artificial intelligence approach based on the blackboard architectural model.
			The architecture relies primarily on machine learning and AI to	solve complex, NP-Hard problems.				
		\end{flushleft}
		
		\textbf{Pros:}
		\begin{itemize}
			\item Made to handle complex, ill-defined problems with the use of AI
			\item Is able to easily adapt to new data
		\end{itemize}
		
		\textbf{Cons:}
		\begin{itemize}
			\item This architecture does not scale well to real world problems
			\item Very complex to design and implement
			\item Requires a large amount of pre existing data to train algorithms
		\end{itemize}
		
		
		
		\newline
		\textbf{- Client-server (3-tier)}
		\newline\newline
		\textbf{Description:}
		
		\begin{flushleft}
			A distributed application structure that partitions tasks between the providers of a resource or service. 
			A three-tier client/server is a type of multi-tier computing architecture in which an entire application 
			is distributed across three different computing layers or tiers. It divides the presentation, application 
			logic and data processing layers across client and server devices. The client will request from the application 
			server and the application server will request from the database server. The client will never communicate 
			without the use of the database server.				
		\end{flushleft}
		
		\textbf{Pros:}
		\begin{itemize}
			\item All files are stored in a central location
			\item Backups and network security is controlled centrally, as only the application server can access the database server
			\item Users can access shared data which is centrally controlled
			\item The workload on the client is lowered with the use of a server				
		\end{itemize}
		
		\textbf{Cons:}
		\begin{itemize}
			\item The server are expensive to purchase and maintain
			\item If any part of the server side network fails then a lot of disruption can occur				
		\end{itemize}
		
		
		
		
		\textbf{- Monolithic application}
		\newline\newline
		\textbf{Description:}
		
		\begin{flushleft}
			A monolithic application is built as a single unit. The application is not separated into smaller partitions and services.
			The single unit involves the program and its own data storage. No centralization exists in this architecture without the use of a server.				
		\end{flushleft}
		
		\textbf{Pros:}
		\begin{itemize}
			\item Simple to design and create
		\end{itemize}
		
		\textbf{Cons:}
		\begin{itemize}
			\item Designed without modularity, i.e. Cannot add to or alter the application without difficulty
			\item Does not scale well for larger applications
			\item Can be difficult to debug, as problems can be difficult to locate on large scale applications
		\end{itemize}
		
		
		
		
		\textbf{- Peer-to-peer (P2P)}
		\newline\newline
		\textbf{Description:}
		
		\begin{flushleft}
			An architecture that partitions tasks or workloads between peers or nodes, where no centralized node exists.
			Data is shared amongst all the nodes and it is easy to remove or insert nodes into the system without 
			affecting the system.
			\newline\newline\newline
			
		\end{flushleft}
		\newline\newline
		\textbf{Pros:}
		\begin{itemize}
			\item More reliable as central dependency is eliminated
			\item The over-all cost of building and maintaining this architecture is comparatively very less than client-server
		\end{itemize}
		\newline\newline
		\textbf{Cons:}
		\begin{itemize}
			\item Weaker security over data than that of client-server
			\item Data recovery or backup is very difficult
			\item Difficult to update applications over this type of architecture
		\end{itemize}
		
		
		
		
		
		
		\newline\newline
		\textbf{- Rule-based Architecture}
		\newline\newline
		\textbf{Description:}
		
		\begin{flushleft}
			The use of forward and backward chaining in order to determine rules that define the architecture of a system.
			Forward chaining observes the given data and makes decisions based on that. Backward chaining observes the goals
			and uses that to make decisions.				
		\end{flushleft}
		
		\textbf{Pros:}
		\begin{itemize}
			\item Easy to implement by defining separate rules in separate layers.
			\item Allows for modularity
			\item All "knowledge" in the architecture is uniform as it can be expressed in the same format defined by the rules
		\end{itemize}
		
		\textbf{Cons:}
		\begin{itemize}
			\item Limited amount of rules that can be defined.
			\item Complexity increases as the system size increases.	\newline\newline			
		\end{itemize}
		
		
		
		
		
		
		
		
		
		\newline\newline
		
		\item \textbf{Choice of architecture and why:} 3-tier Client-server Architecture
		\begin{flushleft}
			
			This architecture provides for an easy, separate development of the back-end and front-end of the application.
			Data will be centralized on the server that will be used and this will provide for strong layer of user privacy
			and security. 
			
			Modularity is easy as alterations to the application can be rolled out on the server side towards
			all clients at once. This architecture is best for this scale of application, and a lot of processing will be done
			server side so less workload will be put on the client. 
			
		\end{flushleft}
	\end{enumerate}
	
\end{document}