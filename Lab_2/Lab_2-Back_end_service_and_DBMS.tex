\documentclass[12pt]{article}
\usepackage{graphicx}
\graphicspath{{./}}

\begin{document}

	\title{COMS3002 - Software Engineering}
	\author{Seale Rapolai (1098005)}
	\maketitle

	\section{Back-end Service and DBMS}
		\begin{enumerate}
			\item \textbf{Back-end Service:} NodeJS
			\begin{flushleft}
				NodeJS is a platform build using Javascript. It is easy to learn and allows developers to build fast and scalable network applications. It uses an event driven, non-blocking I/O model that makes it lightweight and very efficient. This makes it suited for data-intensive and real-time applications that have have high network traffic. It is open source and cross platform, which makes it very portable and versatile. It also has a vast module library which simplifies the development of applications.
				
			\end{flushleft}
			\textbf{Advantages:}
			\begin{itemize}
				\item \textbf{Asynchronous and Event Driven} - NodeJs APIs are asynchronous (non-blocking). This means the server never waits for API to return data before it has to handle other requests.
				\item \textbf{Fast} - It was build on Google Chrome's V8 JavaScript Engine and as such has fast code execution.
				\item \textbf{Single Threaded but Highly Scalable} - It is sigle threaded with event looping. This helps the server to respond in a non-blocking way since each request is handled on after the other. This makes server very scalable as compared to other servers that have limited number of threads to handle requests.
				\item \textbf{No Buffering} - A NodeJS server never buffers data, but instead outputs data in chunks.
			\end{itemize}
			
			\item \textbf{DBMS:} MongoDB
			\begin{flushleft}
				MongoDB is a NoSQL database developed to deal with the limitations of SQL databases which include scalability, multi-structured data, goe-distribution and agile developments sprints. It stores data in flexible, JSON-like documents where fields can vary from document to document. This implies that the data structure can change over time. The document model maps to objects in application code and makes data easy to work with. MongoDB uses Ad hoc queries, indexing and real-time aggregation to provide powerful ways to access and analyse data and it is free and open source.
			\end{flushleft}
		\end{enumerate}

\end{document}
